\section{Usage}

Make sure to have JuMP and Cbc installed before running the program.\newline
The program "main.jl" can take two types of arguments. When the program awaits a user input,
he can specify either an integer or the path to a .txt file of a specific format.
If the user enter an integer n, a random matrix of the size n*n will be generated.
If the user enter the path to the .txt, the matrix it contains will be used.\newline

The .txt must have a specific format. It must be a matrix of the distance between the cities for this shape:\newline

\begin{tabular}{|c|c|c|c|}
  0 & 1 & ... & n \\
  1 & 0 & x & x \\
  ... & x & 0 & x \\
  n & x & x & 0 \\
\end{tabular}
\bigbreak

("data\_test.txt" is provided as an example of the correct format)\newline
Randomly generated matrix are printed on the correct format which means the user can paste them in a
.txt file to save them and reuse them later.\newline

The matrix is printed by default, it can be removed by commenting
"Base.showarray(STDOUT, c, false)" (line 122)\newline
The user can switch between Miller, Tucker, and Zemlin and Finke, Claus, and Gunn by commenting the
lines 128 and 129 depending on
the one he wants to use.
\bigbreak
mode = "MTZ"\newline
\#mode = "FCG"
\bigbreak

The output print on the standard output.
It contains the cycle that show the order the cities are visited, the length of the path that should be the same size as
the matrix. Finally, the objective value. The user can display the result matrix by removing the comment
"Base.showarray(STDOUT, c, false)" (line 53) which is commented by default.

\section{Benchmark}

Benchmark on three randomly-generated matrices of size 5, 10 and 15 respectively\newline
(see bench1.txt, bench2.txt, bench3.txt)

\bigbreak
Results of in seconds (using @time on solve\_tsp)\newline
\begin{tabular}{|c|c|c|}
\ & MTZ & FCG \\
bench1 & 0.68s & 0.64s \\
bench2 & 1.74s & 2.80s \\
bench3 & 72 & 854 \\
\end{tabular}

\bigbreak

Results of the objective value\newline
\begin{tabular}{|c|c|c|}
\ & MTZ & FCG \\
bench1 & 32 & 32 \\
bench2 & 46 & 46 \\
bench3 & 48 & 44 \\
\end{tabular}
\bigbreak
We were able to use MTZ up to a 17*17 matrix with an average solve time of 168s (3 tries).
